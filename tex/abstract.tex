%# -*- coding: utf-8-unix -*-
%%==================================================
%% abstract.tex for SJTU Master Thesis
%%==================================================

\begin{abstract}

CEPC和国际直线对撞机(International Linear Collider,ILC)[10]相仿,都主要利用希格斯韧致辐射(Higgstrahlung,ZH)过程产生希格斯粒子。相比于 LHC 上的质子-质子对撞,正负电子对撞机具有极低的本底,可以提供提供干净的希格斯事例,是精确研究希格斯粒子性质的理想场所。CEPC 的另一巨大优势是可以利用 Z 玻色子反冲质量法来研究希 格斯粒子,只需测量Z玻色子的衰变产物即可探测希格斯粒子。这种方法可以实现模型无关的精确测量,得到希格斯粒子的绝对衰变分支比和耦合常数,与 LHC 的实验结果合并后可以获得最优的结果。反冲质量法还可用于研究末态包含无法直接观测粒子的希格斯衰变道,从而寻找希格斯衰变中可能出现的暗物质和奇异新粒子等。


\keywords{\large 中国未来环形对撞机 \quad 希格斯粒子 \quad 衰变分支比}
\end{abstract}

\begin{englishabstract}

Precise measurement of the Higgs boson properties are important issues for the Circular Electron-Positron Collider(CEPC) project to understand the particles mass generation mechanism which strongly related to the coupling with the Higgs boson. CEPC experiments exclude the large area of the predicted Higgs mass region and their results indicate that Higgs boson mass will be light. 

Even if LHC discovers the Higgs like particle by the end of 2012, Higgs will be identified by the high precision measurement of the Higgs boson properties in CEPC and also Higgs measurement verifies the correctness of standard model (SM) or gives some hints toward its beyond. 

In this study, we evaluate the measurement accuracies of Higgs branching fraction to the H $\to$ $b\bar{b}$, $c\bar{c}$ and $gg$ at the center-of-mass energy of 250 GeV. Only the Monte Carlo samples are used in our analysis.


\englishkeywords{\large CEPC, Higgs Bosons, Higgs branching fraction}
\end{englishabstract}

